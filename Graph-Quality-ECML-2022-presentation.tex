\documentclass[10pt]{beamer}

\usepackage{polyglossia}
\usepackage{csquotes}
\usepackage{datetime}
\usepackage{fontspec}
\usepackage{microtype}
\usepackage{color}
\usepackage{url}
\usepackage{hyperref}
\usepackage{amsfonts}
\usepackage{amsmath}
\usepackage{amsthm}
\usepackage{subcaption}
\usepackage[backend=biber,style=iso-authoryear,sortlocale=en_US,autolang=other,bibencoding=UTF8]{biblatex}
\usepackage{booktabs}
\usepackage{graphics}
\usepackage{pifont}
\usepackage{ulem}
\usepackage{tikz}

\addbibresource{zotero.bib}

\setdefaultlanguage{english}
\setmainfont{TeX Gyre Termes}
\usetheme{Boadilla}
\usecolortheme{crane}
\setbeamertemplate{title page}[default][rounded=true,shadow=false]
\setbeamertemplate{section in toc}[ball unnumbered]
\setbeamertemplate{bibliography item}{}

\hypersetup{
	pdfencoding=auto,
	unicode=true,
	citecolor=green,
	filecolor=blue,
	linkcolor=red,
	urlcolor=blue
}

\makeatletter
\newcommand*{\currentSection}{\@currentlabelname}
\makeatother

\newcommand{\mathmat}{\ensuremath{\mathbf}}

\title[Graph coarsening \& quality]
{
	Adaptive graph coarsening in the context of local graph quality
}

\newdate{presentation}{16}{09}{2022}
\date[September 2022]{Graph Quality 2022, \displaydate{presentation}}

\author[Marek Dědič]
{
	\underline{Marek~Dědič}\inst{1}\inst{2},
	Lukáš~Bajer\inst{2},
	Jakub~Repický\inst{2},
	Pavel~Procházka\inst{2},
	Martin~Holeňa\inst{3}
}

\institute[CTU \& Cisco]
{
	\inst{1} Czech Technical University in Prague \and
	\inst{2} Cisco Systems, Inc. \and
	\inst{3} Institute of Computer Science, Czech Academy of Sciences
}

% Title card
\AtBeginSection[]{
	\begin{frame}
		\vfill
		\centering
		\begin{beamercolorbox}[sep = 8pt,center,shadow = true,rounded = true]{title}
			\usebeamerfont{title}\insertsectionhead\par%
		\end{beamercolorbox}
		\vfill
	\end{frame}
}

%\AtBeginSection[]{
	%\begin{frame}{\currentSection}
		%\tableofcontents[currentsection]
	%\end{frame}
%}

\begin{document}

\begin{frame}
	\titlepage
	\hfill\usebeamerfont{date}\url{marek@dedic.eu}\hspace{1cm}
	\vspace{0.5cm}
\end{frame}

\section{Problem introduction}

\begin{frame}{Problem introduction}
	\begin{itemize}
		\item Graph algorithms on big graphs -- computationally costly and sometimes even unfeasible.
		\item Our interest: Behaviour of graph algorithms when the graph is coarsened
		\item How much can a graph be coarsened?
		\item Sometimes compressed data \( \implies \) better performace
	\end{itemize}
\end{frame}

\section{Prior art - the HARP method}

\begin{frame}{HARP - learning on coarser graphs}
	\begin{itemize}
		\item HARP - a method for pretraining on simplified graphs
		\item Simplified graphs as depicted bellow
		\item Embedding trained first on the coarser graph
	\end{itemize}
	\begin{figure}
		\centering
		\begin{subfigure}[t]{0.38\textwidth}
			\centering
			\includegraphics[width=\textwidth]{images/edge_collapsing.png}
			\caption{Edge collapsing}
		\end{subfigure}
		\hspace{2em}
		\begin{subfigure}[t]{0.38\textwidth}
			\centering
			\includegraphics[width=\textwidth]{images/star_collapsing.png}
			\caption{Star collapsing}
		\end{subfigure}
		\caption{HARP coarsening algorithm. \footnote{Images from \cite{chen_harp_2018}.}}
	\end{figure}
\end{frame}

\begin{frame}{HARP pipeline overview}
	\begin{figure}
		\centering
		\includegraphics[width=\textwidth]{images/harp-overview/harp-overview.pdf}
	\end{figure}
\end{frame}

\section{Adaptive prolongation}

\begin{frame}{HARP pipeline overview}
	\begin{figure}
		\centering
		\includegraphics[width=\textwidth]{images/harp-overview/harp-overview.pdf}
	\end{figure}
\end{frame}

\begin{frame}{Deep HARP}
	\begin{figure}
		\centering
		\includegraphics[width=0.6\textwidth]{images/deep-harp/deep-harp.pdf}
	\end{figure}
\end{frame}

\begin{frame}{Adaptive prolongation}
	\begin{itemize}
		\item Graph prolongation (de-coarsening) micro-stepping
		\item Guiding the prolongation based on local properties
		\item Possibly results in different \enquote{model resolution} in different parts of the graph
		\item Evaluation -- Training a classifier in each (micro) step
	\end{itemize}
\end{frame}

\begin{frame}{Adaptive prolongation}
	\begin{figure}
		\centering
		\includegraphics[width=0.8\textwidth]{images/adaptive-prolongation/adaptive-prolongation.pdf}
	\end{figure}
\end{frame}

\begin{frame}{Adaptive prolongation - experimental verification}
	\begin{figure}
		\centering
		\includegraphics[width=\textwidth]{images/adaptive-coarsening/adaptive-coarsening.pdf}
		\caption{Prediction accuracy with guided prolongation.}
	\end{figure}
\end{frame}

\section{Alternative approaches to coarsening}

\begin{frame}{HARP coarsening}
	\begin{figure}
		\centering
		\begin{subfigure}[t]{0.38\textwidth}
			\centering
			\includegraphics[width=\textwidth]{images/edge_collapsing.png}
			\caption{Edge collapsing}
		\end{subfigure}
		\hspace{2em}
		\begin{subfigure}[t]{0.38\textwidth}
			\centering
			\includegraphics[width=\textwidth]{images/star_collapsing.png}
			\caption{Star collapsing}
		\end{subfigure}
		\caption{HARP coarsening algorithm. \footnote{Images from \cite{chen_harp_2018}.}}
	\end{figure}
\end{frame}

\begin{frame}{Evaluated approaches to coarsening}
	\begin{itemize}
		\item Original HARP
		\item Coarsening based on the personalized PageRank algorithm
		\item Coarsening based on heat diffusion
		\item Evolutionary optimization of coarsening
	\end{itemize}
\end{frame}

\begin{frame}{Experimental comparison of coarsening approaches}
	\begin{figure}
		\centering
		\includegraphics[width=\textwidth]{images/coarsening-algorithms/coarsening-algorithms.pdf}
		\caption{Prediction accuracy with different algorithms.}
	\end{figure}
\end{frame}

\section{Conclusion}

\begin{frame}{Conclusion}
	\centering
	\begin{itemize}
		\item Studying GNN behaviour while coarsening a graph
		\item The adaptive prolongation approach -- a way of studying such behaviour in detail.
		\item 3 alternative coarsening strategies
		\item A new re-formalization of HARP, allowing for a more general approach to graph coarsening
		\item HARP as a framework for graph reduction
	\end{itemize}
\end{frame}

\begin{frame}{Future work}
	\centering
	\begin{itemize}
		\item Investigate more datasets
		\item Surrogate metric for adaptive prolongation
		\item Direct adaptive coarsening (P. Procházka, M. Mareš, M. Dědič, Downstream Task Aware Scalable Graph Size Reduction for Efficient GNN Application on Big Data)
		\item More thorough exploration of evolved coarsenings
	\end{itemize}
\end{frame}

\begin{frame}
	\titlepage
	\hfill\usebeamerfont{date}\url{marek@dedic.eu}\hspace{1cm}
	\vspace{0.5cm}
\end{frame}

\end{document}
