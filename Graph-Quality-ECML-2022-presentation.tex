\documentclass[10pt]{beamer}

\usepackage{polyglossia}
\usepackage{csquotes}
\usepackage{datetime}
\usepackage{fontspec}
\usepackage{microtype}
\usepackage{color}
\usepackage{url}
\usepackage{hyperref}
\usepackage{amsfonts}
\usepackage{amsmath}
\usepackage{amsthm}
\usepackage{subcaption}
\usepackage[backend=biber,style=iso-authoryear,sortlocale=en_US,autolang=other,bibencoding=UTF8]{biblatex}
\usepackage{booktabs}
\usepackage{graphics}
\usepackage{pifont}
\usepackage{ulem}
\usepackage{tikz}

\addbibresource{zotero.bib}

\setdefaultlanguage{english}
\setmainfont{TeX Gyre Termes}
\usetheme{Boadilla}
\usecolortheme{crane}
\setbeamertemplate{title page}[default][rounded=true,shadow=false]
\setbeamertemplate{section in toc}[ball unnumbered]
\setbeamertemplate{bibliography item}{}

\hypersetup{
	pdfencoding=auto,
	unicode=true,
	citecolor=green,
	filecolor=blue,
	linkcolor=red,
	urlcolor=blue
}

\makeatletter
\newcommand*{\currentSection}{\@currentlabelname}
\makeatother

\newcommand{\mathmat}{\ensuremath{\mathbf}}

\title[Q2\_3 Quarterly research review]
{
	Balancing performance and complexity in hierarchical coarsening of graphs.
}

\newdate{presentation}{03}{06}{2022}
\date[June 2022]{\displaydate{presentation}}

\author[Marek Dědič]
{
	Marek~Dědič
}

\institute[Cisco]{}

% Title card
\AtBeginSection[]{
	\begin{frame}
	\vfill
	\centering
	\begin{beamercolorbox}[sep = 8pt,center,shadow = true,rounded = true]{title}
		\usebeamerfont{title}\insertsectionhead\par%
	\end{beamercolorbox}
	\vfill
\end{frame}
}

\begin{document}

\begin{frame}
	\titlepage
\end{frame}

\section{Problem introduction}

\begin{frame}{Node classification}
	\begin{figure}
		\centering
		\includegraphics[width=0.8\textwidth]{images/node_classification.png}\footnote{\cite{kubara_machine_2020}}
	\end{figure}
\end{frame}

\begin{frame}{Problem introduction}
	\begin{itemize}
		\item Graph algorithms on big graphs computationally costly and sometimes even unfeasible.
		\item Our topic: behaviour of graph algorithms when the graph is coarsened
		\item How much can a graph be coarsened?
		\item Sometimes even compressed data \( \implies \) better performace
	\end{itemize}
\end{frame}

\begin{frame}{Pipeline overview}
	\begin{figure}
		\centering
		\includegraphics[width=\textwidth]{images/pipeline_overview/pipeline_overview.pdf}
	\end{figure}
\end{frame}

\section{Prior art}

\begin{frame}{HARP - learning on coarser graphs}
	\begin{itemize}
		\item HARP - a method for pretraining on simplified graphs
		\item Simplified graphs generated in the way depicted bellow
		\item Embedding trained from the coarsest to the finest graph
	\end{itemize}
	\begin{figure}
		\centering
		\begin{subfigure}[t]{0.38\textwidth}
			\centering
			\includegraphics[width=\textwidth]{images/edge_collapsing.png}
			\caption{Edge collapsing}
		\end{subfigure}
		\hspace{2em}
		\begin{subfigure}[t]{0.38\textwidth}
			\centering
			\includegraphics[width=\textwidth]{images/star_collapsing.png}
			\caption{Star collapsing}
		\end{subfigure}
		\caption{HARP coarsening algorithm. \footnote{Images from \cite{chen_harp_2018}.}}
	\end{figure}
\end{frame}

\begin{frame}{Partially injective graph homomorphisms}
	\begin{itemize}
		\item Graph homomorphisms (i. e. subgraph isomorphism)
\[ uv \in E \left( G \right) \implies \varphi \left( u \right) \varphi \left( v \right) \in E \left( H \right) \]
		\item Injective graph homomorphisms (i. e. subgraph isomorphism)
\[ \forall u, v \in V \left( G \right) \quad \varphi \left( u \right) = \varphi \left( v \right) \implies u = v \]
		\item Partially injective graph homomorphisms
\[ \forall uv \in \mathcal{C} \quad \varphi \left( u \right) = \varphi \left( v \right) \implies u = v \]
			where \( \mathcal{C} \subseteq \left( V \left( G \right) \right)^2 \setminus E \left( G \right) \)
	\end{itemize}
\end{frame}

\section{Our work}

\begin{frame}{How do these 2 concepts relate?}
	\begin{itemize}
		\item HARP transformations are P. I. homomorphisms
		\item The set of all partially injective homomorphisms between 2 graphs forms a lattice, allowing for searching for the coarsening operation tailored to the problem at hand
	\end{itemize}
\end{frame}

\begin{frame}{Experimental verification of HARP based on PIHom}
	\begin{figure}
		\centering
		\includegraphics[width=0.8\textwidth]{images/pihom_comparison/pihom_comparison.pdf}
		\caption{Prediction accuracy on OGBN-arxiv}
	\end{figure}
\end{frame}

\begin{frame}{HARP as a complexity-performance trade-off}
	\begin{itemize}
		\item Graph prolongation (de-coarsening) micro-stepping
		\item Guiding the prolongation based on local properties
		\item Possibly results in different \enquote{model resolution} in different parts of the graph
	\end{itemize}
\end{frame}

\begin{frame}{HARP as a complexity-performance trade-off}
	\begin{figure}
		\centering
		\includegraphics[width=0.9\textwidth]{images/accuracy-size.png}
		\caption{Prediction accuracy with guided prolongation}
	\end{figure}
\end{frame}

\begin{frame}{Coarsening models}
	\begin{itemize}
		\item Original HARP
		\item PIHom HARP
		\item Coarsening based on graph diffusion
		\item Evolutionary optimization of coarsening
	\end{itemize}
\end{frame}

\section{Conclusion}

\begin{frame}{Conclusion}
	\centering
	\begin{itemize}
		\item Studying GNN behaviour while coarsening a graph
		\item A new re-formalization of HARP, allowing for a more general approach to graph coarsening
		\item 3 new coarsening strategies
		\item Hopefully a paper in sight
	\end{itemize}
\end{frame}

\begin{frame}
	\centering
	\includegraphics[width=0.75\textwidth]{images/thats_all.png}
\end{frame}

\end{document}
